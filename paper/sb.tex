\documentclass{article} % For LaTeX2e
\usepackage{nips12submit_e,times}
%\documentstyle[nips12submit_09,times,art10]{article} % For LaTeX 2.09

\usepackage{amsmath,amssymb,amsthm,amsfonts,comment}
\usepackage{graphicx}\graphicspath{{figures/}}
\usepackage[small,labelfont=bf]{caption}
\usepackage[square]{natbib}
\usepackage{color}
\usepackage{epstopdf}
\newcommand{\theHalgorithm}{\arabic{algorithm}}

%% our math environment
\def\[#1\]{\begin{align}#1\end{align}}

\newcommand{\defn}[1]{{\bf #1}}

\newcommand{\defas}{:=}
\newcommand{\given}{\mid}

\newcommand{\Naturals}{\mathbb{N}}
\newcommand{\Rationals}{\mathbb{Q}}
\newcommand{\Reals}{\mathbb{R}}
\newcommand{\cS}{\mathcal{S}}
\newcommand{\st}{\,:\,}
\newcommand{\RInts}{\mathcal{I}_\Rationals}
\newcommand{\BSets}{\mathcal{B}_\Reals}
\newcommand{\grad}{\bigtriangledown}

%% our math environment
 
\newtheorem{thm}{Theorem}
\newtheorem{cor}[thm]{Corollary}
\newtheorem{lem}[thm]{Lemma}
%\newtheorem{definition}{Definition}

\title{Learning optimal independence tests for Bayesian Networks}

\author{
\And
Coauthor \\
Affiliation \\
Address \\
\texttt{email} \\
\AND
Coauthor \\
Affiliation \\
Address \\
\texttt{email} \\
\And
Coauthor \\
Affiliation \\
Address \\
\texttt{email} \\
\And
Coauthor \\
Affiliation \\
Address \\
\texttt{email} \\
(if needed)\\
}

% The \author macro works with any number of authors. There are two commands
% used to separate the names and addresses of multiple authors: \And and \AND.
%
% Using \And between authors leaves it to \LaTeX{} to determine where to break
% the lines. Using \AND forces a linebreak at that point. So, if \LaTeX{}
% puts 3 of 4 authors names on the first line, and the last on the second
% line, try using \AND instead of \And before the third author name.

\newcommand{\fix}{\marginpar{FIX}}
\newcommand{\new}{\marginpar{NEW}}

%\nipsfinalcopy % Uncomment for camera-ready version

\begin{document}


\maketitle

\begin{abstract}

\end{abstract}


\section{Introduction}
Learning Bayesian networks for general distributions is
intractable task \cite{chickering1996learning}. However, in practise 
for distributions appearing in real data, still we are able to 
recover Bayesian network structure. This implies that real data
distribution has some special properties, which simplify process
of structure recovery. This process is almost always
based on computation local statistics, and then 
reasoning about global structure \cite{jaakkola2010learning, tsamardinos2006max}. 
Local statistics describe
complexity (e.g. number of parameters in case of BIC), and 
level of independence between nodes (e.g. mutual information tests, 
conditional independence tests). 
We focus in this work on improvement of independence tests by 
learning it for CPDs present in data.



We consider parameterized family of independence tests. 
Before inferring about structure of Bayesian network, we tune classifier 
predict independence or dependence on CPDs present in data. 
This way, we obtain highly sensitive independence test, which
fires on dependence / independence phenomenas present in our data. 


\section{Related work}
Existing independence tests:
\begin{itemize}
\item Pearson's $\chi$-squared.  The problem is the null hypothesis is independence, but independence is what we're trying to show.
\item
\end{itemize}

There has been extensive research in area of scoring functions for
Bayesian networks. This step is critical to recover no-complete graph structure.
Without any scoring 


LL (Log-likelihood) (1912-22)
MDL/BIC (Minimum description length/Bayesian Information Criterion) (1978)
AIC (Akaike Information Criterion) (1974)
NML (Normalized Minimum Likelihood) (2008)
MIT (Mutual Information Tests) (2006)

\cite{margaritis2003learning}

\section{Independence testing}

\begin{figure}[h]
\centering
\includegraphics[width=0.55\linewidth]{img/independence_surface.eps}
\caption{A}
\end{figure}


\subsection{Discrete variables formulation}

\subsection{Continuous variables formulation}

\section{Experiments}

\subsection{Classification of Synthetic CPDs}

\subsection{Classification of CPDs from Gene Expression}

\subsection{Synthetic Bayesian networks}

\subsection{Gene expression data}

\section{Discussion}

\begin{small}
%\renewcommand\bibname{References}
\bibliographystyle{abbrvnat}
%\bibliographystyle{authordate1}
%\bibliographystyle{amsnomr}
\bibliography{bibliography}
\end{small}

%\appendix
%\include{appendix}

\end{document}
